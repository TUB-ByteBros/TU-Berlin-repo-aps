% !TEX program = lualatex
\documentclass[12pt,a4paper]{report} % oder book

\usepackage{fontspec}    % bei lualatex/xelatex
\usepackage{microtype}
\usepackage{graphicx}
\usepackage{amsmath,amssymb}
\usepackage[ngerman]{babel}       % deutsche Silbentrennung
\usepackage[hidelinks]{hyperref}
\usepackage{bookmark}    % besseres TOC → hyperref
\usepackage{fancyhdr}

% Versuche zuerst die gewünschte Systemschrift, sonst Fallback
\IfFontExistsTF{CaskaydiaCove Nerd Font Propo}
  {\setmainfont{CaskaydiaCove Nerd Font Propo}}
  {\setmainfont{Latin Modern Roman}} % Fallback

% Fancyhdr: Seitenzahlen rechts in Fußzeile (du kannst C->zentriert ändern)
\pagestyle{fancy}
\fancyhf{}                        
\fancyfoot[R]{\thepage}           
\renewcommand{\headrulewidth}{0pt} 
\renewcommand{\footrulewidth}{0pt}

% Metadaten für PDF
\hypersetup{
  pdftitle={Titel der Arbeit},
  pdfauthor={Fabian Aps},
  pdfsubject={Thema},
  pdfcreator={LuaLaTeX}
}

\begin{document}

% ------------------
% Titelseite (keine Seitenzahl)
% ------------------
\begin{titlepage}
  \centering
  {\Large Technische Universität Berlin\par}
  \vspace{2cm}
  {\Huge\bfseries Titel der Arbeit\par}
  \vspace{1.5cm}
  {\Large Untertitel (falls vorhanden)\par}
  \vfill
  {\large Autor: Fabian Aps\par}
  {\large Matrikelnummer: 123456\par}
  {\large Studium: B.Sc. Informatik\par}
  \vspace{1cm}
  {\large Betreuer: Prof. Dr. X\par}
  \vspace{2cm}
  {\large \today\par}
  \thispagestyle{empty} % Titelseite ohne Seitenzahl
\end{titlepage}

% ------------------
% Inhaltsverzeichnis mit römischen Seitenzahlen (i, ii, ...)
% ------------------
\cleardoublepage
\pagenumbering{arabic} % 1, 2, 3 ...
\setcounter{page}{1}  % falls gewünscht

\tableofcontents
\clearpage

% ------------------
% Hauptteil: arabische Seitenzahlen ab 1
% ------------------
\cleardoublepage
\pagenumbering{arabic}
\setcounter{page}{2}

\chapter{Einleitung}
Hier beginnt der Hauptteil. Seitenzahlen erscheinen nun als 1, 2, 3, ...

\section{Beispiel}
Text...

\end{document}
